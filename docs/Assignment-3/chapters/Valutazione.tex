\documentclass[../Assignment-3-LPSMT.tex]{subfiles}
\graphicspath{{\subfix{../img/}}}

\begin{document}

\chapter{Valutazione}

Abbiamo intervistato un tester della nostra applicazione e gli abbiamo chiesto di redigere una breve recensione della nostra applicazione.\\
Il tester non è stato scelto a caso, infatti è un appassionato di lettura di fumetti, anche in formato digitale.

\section{Review}

Ho trovato la fase di ``selezione'' dei fumetti leggermente confusionaria, rispetto applicazioni più complesse che si limitano a scannerizzare l'intero contenuto di un folder.\\
Le tempistiche di apertura e caricamento dei documenti sono relativamente rapide.\\
Ho purtroppo trovato scomode le gesture: avrei preferito avanzare il numero della pagina premendo sui lati del telefono, piuttosto che su un soft-button dedicato (per intenderci, la freccia).\\
Siccome è bloccata la rotazione del telefono, considero sgradevole il fitting in larghezza delle splash-page, larghe il doppio rispetto le singole tavole. Per questo motivo, risulta spesso difficile leggere i balloon più piccoli.\\
Avrei preferito fosse presente un disclaimer sull'occupazione di memoria: l'applicazione infatti copia i documenti selezionati presso un suo spazio dedicato in memoria. ``Duplicare'' anche solo 10 volumi può richiedere un enorme spazio su massa.\\
In un futuro aggiornamento, vorrei fosse introdotta la modalità di lettura ``endless'', i.e.\ la possibilità di leggere i fumetti spostandosi con ``swipe'' verso l'alto (in maniera non dissimile alle chat di un'app di messaggistica).

\end{document}
