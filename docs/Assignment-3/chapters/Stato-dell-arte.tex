\documentclass[../Assignment-3-LPSMT.tex]{subfiles}
\graphicspath{{\subfix{../img/}}}

\begin{document}

\chapter{Stato dell'arte}

Le principali applicazioni per la lettura di manga sono 3 ma si basano
tutte su un abbonamento mensile e lettura solo online.

\begin{itemize}
	\item
	      \section{MangaToon} con più di 10 milioni di download e una media di 3.8 stelle su 5.\\
	      L'applicazione è parzialmente gratuita ma manca di alcune traduzioni, una ritardo nella pubblicazione settimanale dei capitoli e nessuna possibilità di mantenere una lista delle letture.\\
	      Il catalogo è basato solo su Manga prodotti da case indipendenti.\\
	      L'app è monotematica per quanto riguarda la demografia delle letture, tutti i titoli sono Shojo~\cite{shooManga}.\\
	      Gli utenti lamentano troppe limitazioni nella versione gratuita e delle pubblicità troppo invasive.\\
	      \href{https://play.google.com/store/apps/details?id=mobi.mangatoon.comics.aphone.spanish}{Pagina del Play Store}.
	\item
	      \section{MANGA Plus}con più di 10 milioni di download e una media di 4.1 stelle su 5.\\
	      L'app parte da un abbonamento gratis che permette di leggere una parte molto ridotta del catalogo.\\
	      Essendo un'applicazione di origine orientale la UI risulta molto diversa dai canoni occidentali, con elementi che distraggono l'occhio e immagini molto grandi e piene di elementi.\\
	      Apparentemente non è possibile effettuare un log in per sincronizzare tra i vari dispositivi l'elenco delle letture.\\
	      Presenta anche una sezione dove gli autori più piccoli possono pubblicare i loro lavori in modo facilitato.\\
	      \href{https://play.google.com/store/apps/details?id=jp.co.shueisha.mangaplus}{Pagina dal Play Store}.
	\item
	      \section{Crunchyroll Manga} con più di 5 milioni di download e una media di 2.8 stelle su 5.\\
	      L'applicazione presenta un abbonamento gratis ed uno premium, con l'abbonamento gratis si ha accesso solo al primo capitolo di alcune opere il che inficia molto sull'esperienza dell'utente.\\
	      A dispetto degli screen sul play store molte opere di grande rilievo non sono presenti e si possono leggere solo Manga indipendenti.\\
	      La UI risulta molto facile da navigare e pulita, anche il reader è molto facile da usare e tiene traccia della pagina alla quale si è arrivati.\\
	     \href{https://play.google.com/store/apps/details?id=com.crunchyroll.crmanga}{Pagina dal Play Store}.
\end{itemize}

La nostra applicazione, come detto, si mette in contrapposizione a questa corrente di mercato fornendo un prodotto gratis ma che si basa sulla lettura offline.\\
L'utente anziché pagare un abbonamento mensile potrebbe acquistare i fumetti in formato digitale e leggerli sulla nostra applicazione.\\
L'incentivo ad usare la nostra applicazione è dato dalla possibilità di gestire le proprie letture in modo simile a come fa la piattaforma
online.\\
All'utente viene quindi tolta la difficoltà di dover navigare attraverso due applicazioni separate per reader e manager.

\end{document}
