\documentclass[../Assignment-3-LPSMT.tex]{subfiles}
\graphicspath{{\subfix{../img/}}}

\begin{document}

\chapter{Analisi critica dei limiti dell’applicazione}

Per quanto strutturata e testata l'applicazione presenta margini di miglioramento.\\
Primo tra tutti lo stile grafico, che non fornisce un'identità propria e purtroppo non è ben integrabile con il nuovo paradigma di colorazione~\cite{matDesColor} basato sul wallpaper, introdotto in Material Design 3.\\
Su suggerimento dei docenti non abbiamo implementato la creazione e la gestione di un account.\\
Il salvare solo l'URI dei file \emph{.cbz} potrebbe portare alla perdita di dati, ma abbiamo preferito questo rispetto all'``esplosione'' dello spazio richiesto dall'app.\\
Per quanto riguarda il reader e la disponibiltà di titoli:
\begin{itemize}
  \item L'implementazione del pintch to zoom nella sezione di reading del manga.
  \item L'attivazione di un opt in per visualizzare anche fumetti per adulti.
  \item L'ampliamento del database per il supporto anche ai comic occidentali.
\end{itemize}

\end{document}
