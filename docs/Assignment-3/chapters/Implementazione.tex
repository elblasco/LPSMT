\documentclass[../Assignment-3-LPSMT.tex]{subfiles}
\graphicspath{{\subfix{../img/}}}

\begin{document}

\chapter{Implementazione}

Manga-check è stata sviluppata seguendo un modello a singola Activity con un
\href{https://developer.android.com/guide/navigation}{controller di navigazione}
che funge da istanziatore e sistema di passaggio di parametri da un fragment ad un altro.\\

\section{Divisione in moduli}

\section{Struttura del DB}

\lstinputlisting[language=Kotlin, caption={Parte di Series.kt}]{snippet-codice/Series.kt}

\lstinputlisting[language=Kotlin, caption={Parte di Chapter.kt}]{snippet-codice/Chapter.kt}

Nella definizione di Chapter abbiamo usato l'attributo \texttt{unique} per poter poi verificare la presenza di numeri duplicati e quindi fermare l'utente dall'inserimento.

\section{Richieste API}

Le richieste API sono state gestite con il sopracitato package \emph{Apollo}, per garantire una fruibilità maggiore tutte le richieste vengono gestite in un Thread separato rieptto a quello della UI.\\
Per ricevere i dati abbiamo usato dei \hyperref{https://developer.android.com/reference/kotlin/androidx/lifecycle/LiveData}{LiveData}, quindi una volta che i dati sono effetivamente presenti vengono mostrati a UI, le immagini vengono gestite grazie a \emph{Glide}.\\
Le entry così generate vengono inserite in una RecyclerView con la quale l'utente andrà ad interagire.

\lstinputlisting[language=Kotlin, caption={Parte del codice usato per fare le Query}]{snippet-codice/Query.kt}

\section{Uso di Safe Args}

Nel progetto abbiamo dovuto trasferire alcuni dati tra due fragment, come indicato nella documentazione Android abbiamo deciso di usare il \emph{navigation graph}, quindi vincolando i dati ad avere un determinato tipo.\\
Questo vincolo è stato possibile grazie all'utilizzo del plug in \href{https://developer.android.com/guide/navigation/use-graph/pass-data#Safe-args}{Safe Args} che ci ha permesso di specificare delle \emph{action} con un paylod di dati tipizzati.\\

\begin{center}
  \includegraphics[scale=0.5]{action_navgraph.png}
  \captionof{figure}{Esempio di action con Safe Args}
\end{center}

\section{Backup Del DB}

Utilizzando le \href{https://developer.android.com/guide/topics/data/autobackup}{funzionalità native} di Android abbiamo implementato un sistema di Backup che permette all'utente di spostare la reading list senza bisogno di esportare alcun file.\\

\section{Reader}

I \emph{cbz} vengono prima decompressi in cache, cosi da non occupare troppa RAM, una volta fatto ciò, i file vengono converti durante l'esecuzione in Bitmap ridimensionate per coprire più superficie possibile.\\
Successivamente le Bitmap vengono gestite sempre da \emph{Glide}, questo ci assicura un'esecuzione asincrona e minimizza il codice da sceivere.\\
Abbiamo anche implementato delle variabili per tenere conto della pagina alla quale è arrivato l'utente, queste variabili sono servite anche per la produzione della barra di progresso nella selezione dei capitoli.

\end{document}
